% Created 2018-09-08 Sat 16:20
\documentclass[presentation]{beamer}
\usepackage[utf8]{inputenc}
\usepackage[T1]{fontenc}
\usepackage{fixltx2e}
\usepackage{graphicx}
\usepackage{longtable}
\usepackage{float}
\usepackage{wrapfig}
\usepackage{rotating}
\usepackage[normalem]{ulem}
\usepackage{amsmath}
\usepackage{textcomp}
\usepackage{marvosym}
\usepackage{wasysym}
\usepackage{amssymb}
\usepackage{hyperref}
\tolerance=1000
\author{Anna e só}
\date{\today}
\title{SecondMeeting}
\hypersetup{
  pdfkeywords={},
  pdfsubject={},
  pdfcreator={Emacs 24.5.1 (Org mode 8.2.10)}}
\begin{document}

\maketitle
\tableofcontents



\section{Processo de parceria}
\label{sec-1}
\subsection{Negociações com instituições parceiras}
\label{sec-1-1}
\begin{itemize}
\item Socialização através de pequenas atividades conjuntas (maratonas de edição, treinamentos).
\item Acordos de cooperação podem ser explicitados através de um \href{https://meta.wikimedia.org/wiki/Partnerships_\%26_Resource_Development/Drafting_a_Memorandum_of_Understanding}{Memorando de Acordos}.
\end{itemize}
\subsection{Antes dos envios}
\label{sec-1-2}
\begin{itemize}
\item Definir o estado dos direitos autorais dos itens
\begin{itemize}
\item Certificar-se de que as licenças são compatíveis com os projetos Wikimedia
\item \href{https://commons.wikimedia.org/wiki/Commons:Licensing}{Página com informações sobre a política de licenças da Wikimedia}
\end{itemize}
\item Preparar os dados
\item Checar o que já está presente nos projetos Wikimedia
\end{itemize}
% Emacs 24.5.1 (Org mode 8.2.10)
\end{document}
